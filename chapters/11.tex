\section{PIV: studio della scia di corpi tozzi}
Lo scopo della presente attività è il calcolo dei campi di velocità nella scia di corpi tozzi mediante la tecnica Time Resolved Particle Image Velocimetry (TR-PIV) utilizzando uno smartphone ed una lama di luce di bassa potenza.

\subsection{Descrizione dell'esperimento}
La Particle Image Velocimetry (PIV) è una tecnica ottica non invasiva che consente di valutare campi istantanei di velocità caratterizzati da un elevato numero di vettori velocità.\\\\
Nella sua continua evoluzione, la metodologia PIV ha dato luogo a forme sempre più sofisticate e complete ai fini della determinazione del campo del vettore velocità istantaneo:
\begin{equation*}
    \vec V (t, x, y)
\end{equation*}
La valutazione della velocità in ogni punto del campo discende direttamente dalla definizione di velocità:
\begin{equation*}
    V = \frac{\Delta s}{\Delta t}
\end{equation*}
La tecnica richiede l'immissione di particelle nel campo che consentono la valutazione della velocità della corrente attraverso il calcolo dello spostamento $\Delta s$. L'intervallo di tempo $\Delta t$ è invece impostato sul sistema dall'operatore.\\\\
La tecnica PIV è caratterizzata da una elevata risoluzione spaziale dovuta alle caratteristiche della telecamera utilizzata. Infatti, in relazione alle dimensioni del campo fisico ripreso il numero di vettori calcolati per unità di superficie dell'immagine ripresa risulta essere molto elevato e tale da rendere la misura tendente al puntiforme. Questo consente di descrivere il campo di moto dettagliatamente nello spazio evidenziando accuratamente le regioni caratterizzate da elevati gradienti di velocità.\\\\
La configurazione Time Resolved Particle Image Velocimetry (TR-PIV) associa l'elevata risoluzione spaziale ad una buona risoluzione temporale che consente di campionare le immagine fino a diverse migliaia di fotogrammi al secondo. Rispetto alla risposta in frequenza della tecnica anemometrica a filo caldo e a quella dell'anemometria LDA quella della tecnica PIV è molto più bassa.\\\\
La sorgente laser emette un raggio luminoso, coerente e monocromatico, che viene fatto espandere attraverso un sistema di lenti generando un piano di luce di spessore molto piccolo.\\\\
La tecnica prevede l'inseminazione della corrente da esaminare mediante particelle iniettate a monte o direttamente nel campo di moto attraverso un sistema di inseminazione. Le particelle, attraversando il piano di luce, sono illuminate e riflettono rendendosi visibili alla telecamera, che ne riprende la loro posizione. L'inseminazione deve essere caratterizzata da una densità di particelle elevata, come regola empirica ogni "area di interrogazione" deve contenere come ordine di grandezza dieci particelle.\\\\
La telecamera deve essere posizionata perpendicolarmente al piano illuminato e deve acquisire le immagini PIV, sulle quali sono riportate le tracce delle particelle illuminate. Le immagini sono costituite da un numero di righe di pixel $M$ e da un numero di colonne di pixel $N$, che individuano il numero totale di pixel. Si definisce quindi risoluzione il prodotto tra $N$ e $M$.\\\\
Da un punto di vista matematico, l'immagine è definita da una funzione che descrive l'intensità del livello di grigio nel piano $x$-$y$ dell'immagine ripresa:
\begin{equation*}
    I = f(x,y)
\end{equation*}
Il livello di grigio è indicato in modo discreto, ad ogni pixel viene quindi assegnato un valore numerico compreso tra 0 e $2^n$, dove $n$ è il numero di bit.\\\\
Ogni immagine PIV viene suddivisa in aree di interrogazione $A_N$ tipicamente quadrate e di dimensioni uguali tra loro. In ogni area di interrogazione ricade un certo numero di particelle, per ogni area viene quindi calcolato lo spostamento delle particelle contenute. L'area di interrogazione deve essere sufficientemente piccola in modo tale da poter ritenere il flusso uniforme, non deve essere troppo grande in quanto si rischia di perdere la descrizione dettagliata del campo. Più l'area è grande e meno verificata sarà la condizione di flusso uniforme.\\\\
Contemporaneamente, l'area di interrogazione deve contenerre un numero minimo di particelle (dell'ordine di dieci) affinché sia possibile valutare uno spostamento statistico affidabile attraverso gli algoritmi di calcolo.\\\\
L'analisi delle immagini per la determinazione degli spostamenti delle particelle può essere effettuata in due diversi domini:
\begin{itemize}
    \item Nel dominio dello spazio mediante la valutazione della funzione di autocorrelazione per immagini multi-esposte (poco utilizzata) e della funzione di cross-correlazione nel caso di immagini singolarmente esposte, questa nel continuo è definita come:
    \begin{equation*}
        R_{j,j+1}(r_1,r_2) = \iint_{A_N} F_j(x,y)F_{j+1}(x+r_1,y+r_2)dxdy
    \end{equation*}
    Nel discreto, la funzione di cross-correlazione si calcola come:
    \begin{equation*}
        R_{j,j+1}(r_1,r_2) = \sum_{h=1}^{\Delta h}\sum_{k=1}^{\Delta k} F_j(h,k) F_{j+1}(h+r_1, k+r_2)\Delta h \Delta k
    \end{equation*}
    \item Nel dominio delle frequenze mediante la valutazione della Fast Fourier Transform (FFT) di due immagini successive. È il procedimento operativo che si segue perché risulta del tutto equivalente al procedimento di analisi mediante cross-correlazione ma risulta essere molto più veloce.
\end{itemize}

\noindent La FFT è un algoritmo per calcolare la Trasformata di Fourier Discreta (DFT) in modo efficiente. La DFT di una sequenza $f(x)$ è definita come:
\begin{equation*}
    \mathcal F(k) = \sum_{n=0}^{N-1} f(n) e^{-i 2 \pi k n / N}
\end{equation*}
La cross-correlazione può essere calcolata utilizzando la FFT tramite il teorema di convoluzione:
\begin{equation*}
C(x, y) = \mathcal{F}^{-1} \{ \mathcal{F}[A(x, y)] \cdot \mathcal{F}[B(x, y)]^* \}
\end{equation*}
dove $\mathcal{F}$ denota la trasformata di Fourier e $\mathcal{F}^{-1}$ l'antitrasformata, mentre con l'apice $^*$ si denota il complesso coniugato.\\\\
Mediante questi algoritmi numerici, la cross-correlazione di ciascuna area di interrogazione tra due immagini successive permette di determinare lo spostamento delle particelle e quindi il campo di velocità del fluido.

\subsection{Catena di misura}
L'esperimento viene condotto in un canale idrodinamico a pelo libero, il cui flusso è messo in moto da un'elica spingente azionata da un motore brushless. I corpi analizzati (cilindro o placca piana) sono posizionati nella camera di prova del canale in condizione di bassi regimi di numero di Reynolds. La bassa velocità del flusso permette di utilizzare la telecamera di uno smartphone.
\begin{figure}[H]
    \centering
    \includegraphics[height=.46\textwidth]{images/11/piv1.jpg}
    \includegraphics[height=.46\textwidth]{images/11/piv2.jpg}
    \caption{Canale idrodinamico}
\end{figure}

\noindent Nel caso del cilindro, il flusso che si sviluppa non risulta essere quello canonico 2D per via della sua dimensione finita che appoggia sulla parete inferiore della camera di prova. Inoltre, il piano di misura è a un diametro di distanza dal fondo, dove si sviluppa il vortice a ferro di cavallo (horseshoe vortex) e sono presenti effetti tridimensionali dovuti all'induzione del vortice. Per questi motivi i casi da studiare differiscono dai flussi canonici bidimensionali, pertanto, l'andamento Reynolds-Strouhal potrebbe discostarsi da quello studiato per il cilindro 2D utilizzando l'anemometria a filo caldo.\\\\
I corpi analizzati sono un cilindro (diametro $d=10$ mm) ed una placca piana rettangolare (lunghezza $L=24$ mm).
\begin{figure}[H]
    \centering
    \includegraphics[width=.5\textwidth]{images/11/rapprcorpi.png}
    \caption{Sezione dei corpi tozzi in esame}
\end{figure}

\noindent Per le misure si utilizza un sistema PIV "low cost", costituito dalla telecamera di uno smartphone la cui risoluzione è di 1920$\times$1080 pixel e frequenza di acquisizione delle immagini $f_{samp}$ pari a 60 fotogrammi al secondo, che corrisponde ad un intervallo temporale $\Delta t = 16.7$ ms.
\begin{figure}[H]
    \centering
    \includegraphics[width=.8\textwidth]{images/11/rapprcatena.png}
    \caption{Rappresentazione del setup sperimentale}
\end{figure}

\noindent Le particelle impiegate sono costituite da carbonato di silice ($\rho_p$=1.1 g/cm$^3$, $d_p$=2$\mu$m).\\\\
La sorgente laser ha una potenza di 30 mW e genera un fascio di luce continuo con lunghezza d'onda di 532 nm (colore verde). Una lente cilindrica, montata direttamente nella testa della sorgente laser, diverge il fascio laser lungo una sola direzione creando la lama di luce (laser sheet).\\\\
In questa configurazione viene a mancare il sincronismo tra telecamera e lama di luce, poiché l'emissione della sorgente laser è continua.
\begin{figure}[H]
    \centering
    \includegraphics[width=.7\textwidth]{images/11/catenastrumenti.png}
    \caption{Sorgente laser, smartphone, particelle di poliammide}
\end{figure}

\noindent Le immagini ottenute hanno un fattore di scala $Sc$ [mm/pixel] da determinare, che deve essere usato per la calibrazione geometrica, cioè per trasformare lo spostamento misurato in pixel sulle immagini PIV acquisite in millimetri di spostamento al vero.

\subsection{Procedura sperimentale}
Per acquisire le immagini PIV, prima si mette in moto il fluido, già inseminato con le particelle di carbonato di silice, nel canale idrodinamico mediante il motore brushless che aziona un'elica spingente, facendo attenzione a non incorrere in fenomeni di cavitazione.\\\\
Successivamente si accende la sorgente laser, da cui si genera la lama di luce che attraversa il condotto nella camera di prova illuminando le particelle.\\\\
Poi si posiziona il corpo (cilindro o placca piana) a monte della camera di prova e mediante uno smartphone (che deve essere opportunamente posizionato e stabile) si registra un video di circa 10 secondi ad una risoluzione di 1920$\times$1080 a 60 fotogrammi al secondo.\\\\
Per analizzare le immagini si utilizza PIVlab, un tool box di Matlab che offre un'interfaccia grafica intuitiva per calcolare la distribuzione della velocità nei vari fotogrammi.
\begin{figure}[H]
    \centering
    \includegraphics[width=.4\textwidth]{images/11/pivlabui.png}
    \caption{Interfaccia di avvio di PIVlab}
\end{figure}

\noindent PIVlab permette di importare le immagini, pre-processarle per migliorare il contrasto tra particelle e sfondo (se necessario) e applicare la calibrazione geometrica.
\begin{figure}[H]
    \centering
    \includegraphics[width=\textwidth]{images/11/pivlabcg.png}
    \caption{Calibrazione geometrica in PIVlab (cilindro 2022)}
\end{figure}

\noindent Applicando la calibrazione geometrica si ricava lo scaling factor:
\begin{equation*}
    Sc = \frac{10\text{ mm}}{97.6596\text{ px}} = 0.1024\ \frac{\text{mm}}{\text{px}}
\end{equation*}
Poiché la frequenza di acquisizione dei fotogrammi è pari a 60 Hz ($\Delta t = 16.67$ ms), si può stimare il valore di velocità equivalente allo spostamento di un pixel in due fotogrammi successivi:
\begin{equation*}
    1\   \frac{\text{px}}{\text{frame}} = \frac{0.1024 \text{ mm}}{16.67 \text{ ms}} = 0.0061\ \frac{\text{m}}{\text{s}}
\end{equation*}
Ovviamente il valore dello scaling factor $Sc$ varia a seconda del caso in esame.\\\\
Il toolbox PIVlab comprende gli algoritmi numerici della PIV con cui si può valutare il generico campo di velocità istantaneo per ogni coppia di immagini PIV acquisite. I campi di velocità calcolati da PIVlab vengono infine esportati come file \texttt{.mat}, da fornire in input a Matlab per effettuare l'analisi dati.

\newpage
\subsection{Analisi dati}
L'analisi dati per la presente attività è condotta con l'ausilio di alcuni codici in Matlab, riportati in appendice \ref{b11}.\\\\
% CITARE ANIMAZIONI E PROCEDURA DI ACQUISIZIONE FOTOGRAMMI

% CAMPO ISTANTANEO VELOCITA', CONFRONTO CON FOTOGRAMMA VIDEO (CIL 22, PLA 22)
% CAMPO ISTANTANEO VORTICITA' (CIL 22, PLA 22)

% CALCOLO DEL CAMPO MEDIO DI VELOCITA' (CIL 22, PLA 22)
% CALCOLO DEL CAMPO MEDIO DI VORTICITA' (CIL 22, PLA 22)
% COMMETARE VALORI MASSIMI, ACCELERAZIONE NEL CONVERGENTE ED EVENTUALI VORTICI

% DISCUTERE EVOLUZIONE DELLA VELOCITA' IN UN PUNTO FISSO (CIL 22, PLA 22)
% SPIEGARE PERCHÈ WELCH NON VA BENE
% TRASFORMATA DI FOURIER DISCRETA E SPETTRO DI POTENZA (CIL 22, PLA 22)
% CALCOLO DELLA FREQUENZA DI SHEDDING
% STIMA DELLA VELOCITA' A MONTE

% CALCOLO DEL NUMERO DI REYNOLDS E DEL NUMERO DI STROUHAL (CIL 22, CIL 23, PLA 22)
    % CILINDRO 22
        % Re = 1296 St = 0.215 fs = 2.60 Hz Uinf = 0.121 m/s
    % CILINDRO 23
        % Re = 1225 St = 0.185 fs = 2.10 Hz Uinf = 0.114 m/s
    % PLACCA 22
        % Re = 2331 St = 0.239 fs = 0.9 Hz Uinf = 0.090 m/s
% CONFRONTARE RISULTATI CON ESERCITAZIONE 10 (CIL 22, CIL 23)