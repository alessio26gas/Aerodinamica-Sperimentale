\section{Codici Python}
Nella presente appendice sono riportati i codici Python sviluppati per l'analisi dati delle varie esperienze di laboratorio.

\subsection{Taratura di un trasduttore di pressione}\label{b1}
\begin{lstlisting}
from numpy import *
from matplotlib.pyplot import *


def main():
    p1, E1, rms1 = load_data("Squadra1.txt", ",")
    p2, E2, rms2 = load_data("Squadra2.txt", "\t")
    p3, E3, rms3 = load_data("Squadra3.txt", None)
    E1 -= E1[0]
    E2 -= E2[1]
    E3 -= E3[0]
    E = [E1, E2, E3]
    p = [p1, p2, p3]
    rms = [rms1, rms2, rms3]

    Kt = []
    for i in range(3):
        c = polyfit(E[:][i], p[:][i], 1)
        Kt.append(c[0])
 
    fsamp = 2000
    i0 = 2480
    t0 = i0 / fsamp
    datat = load_trans()
    transitorio = datat[i0:]

    t = linspace(t0, 120 - 1/fsamp, len(datat) - i0)
    e0 = average(datat[:i0-1000]) * ones(i0)
    c = polyfit(t, log(-transitorio), 1)
    b = c[0]
    A = exp(c[1])
    tau = -1/b
    y = -A*exp(-t/tau)
    t = concatenate((linspace(0, t0, i0), t))

    print(f"Costante Kt squadra 1: {Kt[0]:.2f} Pa/V")
    print(f"Costante Kt squadra 2: {Kt[1]:.2f} Pa/V")
    print(f"Costante Kt squadra 3: {Kt[2]:.2f} Pa/V")
    print(f"Tempo caratteristico tau = {tau:.2f} s")

    figure(1)
    scatter(p1, E1)
    scatter(p2, E2)
    scatter(p3, E3)
    errorbar(p1, E1, yerr=rms[0], fmt="o")
    errorbar(p2, E2, yerr=rms[1], fmt="o")
    errorbar(p3, E3, yerr=abs(rms[2]), fmt="o")
    xlabel(r"p$_{Betz}$ [Pa]")
    ylabel("E [V]")
    grid()
    legend(("Squadra 1", "Squadra 2", "Squadra 3"))
    savefig("images/kt", dpi=400)
    close()

    figure(2)
    plot(t, datat, label="Dati sperimentali")
    plot(t, concatenate((e0, y)), label="Curva interpolante")
    xlabel("t [s]")
    ylabel("E [V]")
    grid()
    legend()
    savefig("images/transitorio", dpi=400)
    close()


def load_data(file, sep):
    with open(file, 'r') as f:
        p = []
        E = []
        rms = []
        next(f)
        for line in f:
            values = [float(x) for x in line.strip().split(sep)]
            p.append(values[0])
            E.append(values[1])
            rms.append(values[2])
        p = array(p)
        E = array(E)
        rms = array(rms)
        return p, E, rms


def load_trans():
    with open("Transitorio.txt", 'r') as f:
        E = []
        next(f)
        next(f)
        for line in f:
            E.append(float(line.strip()))
        E = array(E)
        return E
    

if __name__ == "__main__":
    main()
\end{lstlisting}

\subsection{Risposta direzionale di un tubo di Pitot}\label{b2}
\begin{lstlisting}
from numpy import *
from matplotlib.pyplot import *


def main():
    T0 = array([23.3, 24.2, 24.9, 23.9]) + 273.15
    p0 = array([102300, 99400, 99400, 101200])
    rho = p0 / T0 / 287
    mu = 1.45e-6 * T0**1.5 / (110+T0)
    D = 0.003
    Kt = 550

    ps, ptot, q = load_files()

    qv = []
    qv.append(Kt*(q[0][1, 1]-q[0][0, 1]))
    qv.append(Kt*(q[1][1, 1]-q[1][0, 1]))
    qv.append(Kt*(q[2][1, 1]-q[2][0, 1]))
    qv.append(Kt*(q[3][1, 1]-q[3][0, 1]))
    qv = array(qv)

    V = sqrt(2 * qv / rho)
    Re = rho * V * D / mu

    eps, eptot, eq = [], [], []
    for i in range(4):
        eps.append(Kt*(ps[i][1:, 1] - ps[i][1, 1])/qv[i])
        eptot.append(Kt*(ptot[i][1:, 1] - ptot[i][1, 1])/qv[i])
        eq.append(Kt*(q[i][1:, 1] - q[i][1, 1])/qv[i])

    figure(1)
    for i in range(4):
        scatter(ps[i][1:, 0], eps[i], label=f"Sq {i+1}, Re: {Re[i]:.0f}")
    grid()
    title("Static Pressure")
    xlabel(r"$\alpha$ [deg]")
    ylabel(r"$\varepsilon$")
    legend()
    tight_layout()
    savefig("images/p", dpi=400)
    close()

    figure(2)
    for i in range(4):
        scatter(ptot[i][1:, 0], eptot[i], label=f"Sq {i+1}, Re: {Re[i]:.0f}")
    grid()
    title("Total Pressure")
    xlabel(r"$\alpha$ [deg]")
    ylabel(r"$\varepsilon$")
    legend()
    tight_layout()
    savefig("images/ptot", dpi=400)
    close()

    figure(3)
    for i in range(4):
        scatter(q[i][1:, 0], eq[i], label=f"Sq {i+1}, Re: {Re[i]:.0f}")
    grid()
    title("Dynamic Pressure")  
    xlabel(r"$\alpha$ [deg]")
    ylabel(r"$\varepsilon$")
    legend()
    tight_layout()
    savefig("images/q", dpi=400)
    close()

    qt1 = load_trans("Squadra 1/transitorioq.txt")-q[0][0, 1]

    i0, i0f = 1500, 3000
    e0 = average(qt1[:i0-300]) * ones(i0)
    ef = average(qt1[i0f+300:]) * ones(len(qt1)-i0f)
    t = linspace(i0/500, i0f/500, i0f-i0)
    transitorio = qt1[i0:i0f]
    c = polyfit(t, log((transitorio)), 1)
    b = c[0]
    A = exp(c[1])
    tau = -1/b
    y = A*exp(-t/tau)
    y = concatenate((e0, y, ef))
    t = linspace(0, 120-1/500, len(qt1))
    plot(t, qt1, label="Dati sperimentali")
    plot(t, y, label="Curva interpolante")
    xlabel("t [s]")
    ylabel("E [V]")
    grid()
    tight_layout()
    legend()
    savefig("images/transitorio", dpi=400)
    close()

    print(f"Tempo caratteristico: {tau*1000:.2f} ms")


def load_files():
    ps, ptot, q = [], [], []
    ps.append(load_data("Squadra 1/ps.txt"))
    ptot.append(load_data("Squadra 1/ptot.txt"))
    q.append(load_data("Squadra 1/q.txt"))
    ps.append(load_data("Squadra 2/ps.txt"))
    ptot.append(load_data("Squadra 2/ptot.txt"))
    q.append(load_data("Squadra 2/q.txt"))
    ps.append(load_data("Squadra 3/ps.txt"))
    ptot.append(load_data("Squadra 3/ptot.txt"))
    q.append(load_data("Squadra 3/q.txt"))
    ps.append(load_data("Squadra 4/ps.txt"))
    ptot.append(load_data("Squadra 4/ptot.txt"))
    q.append(load_data("Squadra 4/q.txt"))
    return ps, ptot, q


def load_data(file):
    with open(file, 'r') as f:
        a = []
        E = []
        rms = []
        next(f)
        for line in f:
            values = [float(x) for x in line.strip().split()]
            a.append(values[0])
            E.append(values[1])
            rms.append(values[2])
        a = array(a)
        E = array(E)
        rms = array(rms)
        return column_stack((a, E, rms))


def load_trans(file):
    with open(file, 'r') as f:
        E = []
        for line in f:
            E.append(float(line.strip().split()[2]))
        E = array(E)
        return E
    

if __name__ == "__main__":
    main()
\end{lstlisting}

\subsection{Struttura del getto}\label{b3}
\begin{lstlisting}
from numpy import *
from matplotlib.pyplot import *
from glob import glob
from mpl_toolkits.mplot3d import Axes3D
from scipy.interpolate import interp1d


def main():
    Dj = 0.026
    Kt = 550
    p = 99400
    T = 21.9 + 273.15
    rho = p / T / 287
    mu = 1.46E-6 * T**1.5 / (T + 110)

    for sq in range(4):
        file_paths = glob(f"Sq{sq+1}/x*.txt")
        r, E, Erms, x = [], [], [], []
        for file_path in file_paths:
            if file_path == f"Sq{sq+1}/x0E0.txt":
                E0 = loadtxt(file_path)[1]
            else:
                x.append(float(file_path.split('x')[1].split('.t')[0]))
                r_i, E_i, Erms_i = genfromtxt(file_path, skip_header=1).T
                r.append(r_i)
                E.append(E_i)
                Erms.append(Erms_i)
        x, r, E, Erms = zip(*sorted(zip(x, r, E, Erms)))
        x = array(x)
        r, E, Erms = list(r), list(E), list(Erms)
        for i in range(len(E)):
            E[i] -= E0

        if sq == 3: # Effetti spuri squadra 4
            for i in range(len(r)):
                r[i] -= 10/300 * x[i]

        for i in range(len(x)):
            r[i], E[i], Erms[i] = zip(*sorted(zip(r[i], E[i], Erms[i])))
            r[i], E[i], Erms[i] = array(r[i]), array(E[i]), array(Erms[i])
            E[i] = concatenate((E[i][r[i]>0][::-1], E[i][r[i]>=0]))
            Erms[i] = concatenate((Erms[i][r[i]>0][::-1], Erms[i][r[i]>=0]))
            r[i] = concatenate((-r[i][r[i]>0][::-1], r[i][r[i]>=0]))

        U0 = sqrt(2 * max(E[0]) * Kt / rho)
        for i in range(len(x)):
            U = sign(2 * E[i] * Kt / rho)*sqrt(abs(2 * E[i] * Kt / rho))
            Umax = max(U)
            interp_func = interp1d(U[r[i]>0], r[i][r[i]>0], fill_value="extrapolate")
            ru = interp_func(Umax/2)

            figure(0)
            plot(r[i], U, label=f"x={x[i]:.0f} mm")
            figure(1)
            plot(r[i]/Dj/1000, U/U0, label=f"x={x[i]:.0f} mm")
            figure(2)
            plot(r[i][r[i]>=0]/ru, U[r[i]>=0]/Umax, label=f"x={x[i]:.0f} mm")

        figure(0)
        title(f"Squadra {sq+1}")
        xlabel("r [mm]")
        ylabel("u(x,r) [m/s]")
        grid()
        legend()
        tight_layout()
        savefig(f"images/sq{sq+1}", dpi=400)
        close()

        figure(1)
        title(f"Squadra {sq+1}")
        xlabel(r"$\frac{r}{D_j}$")
        ylabel(r"$\frac{u(x,r)}{U_0}$")
        grid()
        legend()
        tight_layout()
        savefig(f"images/sq{sq+1}u0", dpi=400)
        close()

        figure(2)
        title(f"Squadra {sq+1}")
        xlabel(r"$\frac{r}{r_{u=0.5U_{max}}(x)}$")
        ylabel(r"$\frac{u(x,r)}{U_{max}(x)}$")
        grid()
        legend()
        tight_layout()
        savefig(f"images/sq{sq+1}umax", dpi=400)
        close()

        fig = figure(3)
        ax = fig.add_subplot(111, projection='3d')
        for i in range(len(x)):
            _z = sign(2 * E[i] * Kt / rho)*sqrt(abs(2 * E[i] * Kt / rho))
            _z = concatenate((_z[r[i]>=0][::-1], _z[r[i]>=0]))
            _x = concatenate((-r[i][r[i]>=0][::-1], r[i][r[i]>=0]))
            _y = x[i]*ones_like(_x)
            ax.plot(_x, _y, _z)

        ax.set_xlabel('r [mm]')
        ax.set_ylabel('x [mm]')
        ax.set_zlabel('u [m/s]')
        tight_layout()
        savefig(f"images/sq{sq+1}3d", dpi=600)
        close()

        Re = rho * U0 * Dj / mu
        print(f"Reynolds squadra {sq+1}: {Re:.0f}")

        close('all')


if __name__ == "__main__":
    main()
\end{lstlisting}

\subsection{Proprietà del getto}\label{b4}
\begin{lstlisting}
from numpy import *
from matplotlib.pyplot import *
from glob import glob
from scipy.interpolate import interp1d


def main():
    Dj = 0.026
    Dp = 0.003
    Kt = 550
    p = 99400
    T = 21.9 + 273.15
    rho = p / T / 287
    mu = 1.46E-6 * T**1.5 / (T + 110)

    pts = [3, 3, 3, 2]

    for sq in range(4):
        file_paths = glob(f"Sq{sq+1}/x*.txt")
        r, E, Erms, x = [], [], [], []
        for file_path in file_paths:
            if file_path == f"Sq{sq+1}/x0E0.txt":
                E0 = loadtxt(f"Sq{sq+1}/x0E0.txt")[1]
            else:
                x.append(float(file_path.split('x')[1].split('.t')[0]))
                r_i, E_i, Erms_i = genfromtxt(file_path, skip_header=1).T
                r.append(r_i)
                E.append(E_i)
                Erms.append(Erms_i)
        x, r, E, Erms = zip(*sorted(zip(x, r, E, Erms)))
        x = array(x)
        r, E, Erms = list(r), list(E), list(Erms)
        for E_i in E:
            E_i -= E0

        if sq == 3: # Effetti spuri squadra 4
            for i in range(len(r)):
                r[i] -= 10/300 * x[i]

        for i in range(len(x)):
            r[i], E[i], Erms[i] = zip(*sorted(zip(r[i], E[i], Erms[i])))
            r[i], E[i], Erms[i] = array(r[i]), array(E[i]), array(Erms[i])

        Umax, delta = [], []
        G, M, Ec = [], [], []
        U0 = sqrt(2 * max(E[0]) * Kt / rho)
        for i in range(len(x)):
            U = sign(2 * E[i] * Kt / rho)*sqrt(abs(2 * E[i] * Kt / rho))
            Umax.append(max(U))
            interp_func = interp1d(U[r[i]>0], r[i][r[i]>0], fill_value="extrapolate")
            delta.append(interp_func(max(U)/2))
            G.append(trapz(2*pi*rho*U[r[i]>=0]*r[i][r[i]>=0]/1000, r[i][r[i]>=0]/1000))
            M.append(trapz(2*pi*rho*(U[r[i]>=0]**2)*r[i][r[i]>=0]/1000, r[i][r[i]>=0]/1000))
            Ec.append(trapz(pi*rho*(U[r[i]>=0]**3)*r[i][r[i]>=0]/1000, r[i][r[i]>=0]/1000))
        delta = array(delta)

        y_ = log(Umax[-pts[sq]:]/U0)
        x_ = log(x[-pts[sq]:]/Dj/1000)
        c = polyfit(x_, y_, 1)
        m = -c[0]
        k = exp(c[1])
        xx = linspace(.1, 20)
        f = k/(xx**m)
        for i in range(len(f)):
            if f[i] > 1:
                f[i] = 1

        G0 = rho * (pi*Dj**2/4) * U0
        M0 = rho * (pi*Dj**2/4) * U0**2
        Ec0 = 0.5 * rho * (pi*Dj**2/4) * U0**3
        
        Re = rho * U0 * Dj / mu
        print(f"Squadra {sq+1} Re: {Re:.0f} m: {m:.3f} k: {k:.3f}")

        figure(0)
        d, = plot(array(x)/Dj/1000, Umax/U0, "o", label=f"Sq. {sq+1}. Re: {Re:.0f}")
        plot(xx, f, color=d.get_color())
        figure(1)
        plot(array(x)/Dj/1000, delta/Dj/1000, "-o", label=f"Sq. {sq+1}. Re: {Re:.0f}")
        figure(2)
        plot(x, array(G)/G0, label=f"Squadra {sq+1}")
        figure(3)
        plot(x, array(M)/M0, label=f"Squadra {sq+1}")
        figure(4)
        plot(x, array(Ec)/Ec0, label=f"Squadra {sq+1}")

    figure(0)
    title("Velocita' massima")
    xlabel(r"$\frac{x}{D_j}$")
    ylabel(r"$\frac{U_{max}(x)}{U_0}$")
    grid()
    legend()
    savefig("images/umax", dpi=400)

    figure(1)
    title("Dimensione trasversale del getto")
    xlabel(r"$\frac{x}{D_j}$")
    ylabel(r"$\frac{\Delta(x)}{D_j}$")
    grid()
    legend()
    savefig("images/delta", dpi=400)

    figure(2)
    title("Portata")
    xlabel(r"$\frac{x}{D_j}$")
    ylabel(r"$\frac{G(x)}{G_0}$")
    grid()
    legend()
    savefig("images/portata", dpi=400)

    figure(3)
    title("Quantita' di moto")
    xlabel(r"$\frac{x}{D_j}$")
    ylabel(r"$\frac{M(x)}{M_0}$")
    grid()
    axis([-25, 525, 0, 1.25])
    legend()
    savefig("images/qdm", dpi=400)

    figure(4)
    title("Energia cinetica")
    xlabel(r"$\frac{x}{D_j}$")
    ylabel(r"$\frac{E_c(x)}{E_{c0}}$")
    grid()
    legend()
    savefig("images/energia", dpi=400)

    close("all")


if __name__ == "__main__":
    main()
\end{lstlisting}

\subsection{Profilo alare Naca 0015}\label{b5}
\begin{lstlisting}
import csv
from numpy import *
from matplotlib.pyplot import *


def main():
    x, cpd, cpv = import_data()
    a, CL, CMLE, CM, xcp = ([[], [], [], []] for _ in range(5))
    CLa, CMLEa, xac = (zeros(4) for _ in range(3))
    for sq in range(4):
        for k in range(7):
            a[sq].append(cpd[sq][k+1][0])
            CL[sq].append(trapz(array(cpv[sq][k+1][1:])-array(cpd[sq][k+1][1:]), x))
            CMLE[sq].append(trapz(x*(array(cpd[sq][k+1][1:])-array(cpv[sq][k+1][1:])), x))
        xcp[sq] = -array(CMLE[sq][1:])/array(CL[sq][1:])
        CLa[sq] = polyfit(radians(a[sq][:5]), CL[sq][:5], 1)[0]
        CMLEa[sq] = polyfit(radians(a[sq][:3]), CMLE[sq][:3], 1)[0]
        xac[sq] = - CMLEa[sq]/CLa[sq]
        CM[sq] = array(CMLE[sq]) + array(CL[sq]) * 0.25

    save_cp_plots(x, cpd, cpv)
    save_cl_plots(a, CL)
    save_cmle_plots(a, CMLE)
    save_xcp_plots(a, xcp)
    save_cm_plot(a, CM)
    save_cpfp_plot(x, cpd)
    print(f"Posizione relativa del fuoco: {sum(xac)/len(xac):.4f}")

    
def save_cpfp_plot(x, cpd):
    fp = loadtxt(open("cpcsv/Data(fp).csv", "rb"), delimiter=",", skiprows=1)
    xfp = fp[:,0]
    cpfp = fp[:,2]
    title(f"Confronto con teoria del flusso potenziale")
    plot(xfp, cpfp, '-o')
    plot(x, cpd[0][1][1:], '-o')
    xlabel("x/c")
    ylabel("$c_p$")
    gca().invert_yaxis()
    grid()
    legend(["Teoria del flusso potenziale", "Risultati sperimentali"])
    savefig(f'images/cpfp')


def save_cm_plot(a, CM):
    for sq in range(4):
        plot(a[sq], CM[sq],'-o')
    ylim([-0.35, 0.35])
    title(f"Coefficiente di momento aerodinamico")
    xlabel("alpha [deg]")
    ylabel("$C_{M}$")
    grid()
    legend(["Squadra 1", "Squadra 2", "Squadra 3", "Squadra 4"])
    savefig(f'images/cm')
    close()


def save_xcp_plots(a, xcp):
    for sq in range(4):
        plot(a[sq][1:], xcp[sq], '-o')
        title(f"Squadra {sq+1}, Centro di pressione")
        xlabel("alpha [deg]")
        ylabel("$x_{cp}/c$")
        grid()
        savefig(f'images/xcp{sq+1}')
        close()
    for sq in range(4):
        plot(a[sq][1:], xcp[sq], '-o') 
    title(f"Centro di pressione")
    xlabel("alpha [deg]")
    ylabel("$x_{cp}/c$")
    grid()
    legend(["Squadra 1", "Squadra 2", "Squadra 3", "Squadra 4"])
    savefig(f'images/xcp')
    close()


def save_cmle_plots(a, CMLE):
    for sq in range(4):
        plot(a[sq], CMLE[sq], '-o')
        title(f"Squadra {sq+1}, Coefficiente di momento al bordo di attacco")
        xlabel("alpha [deg]")
        ylabel("$c_{M_{LE}}$")
        grid()
        savefig(f'images/cmle{sq+1}')
        close()
    
    for sq in range(4):
        plot(a[sq], CMLE[sq], '-o')
    title(f"Coefficiente di momento al bordo di attacco")
    xlabel("alpha [deg]")
    ylabel("$c_{M_{LE}}$")
    grid()
    legend(["Squadra 1", "Squadra 2", "Squadra 3", "Squadra 4"])
    savefig(f'images/cmle')
    close()


def save_cl_plots(a, CL):
    for sq in range(4):
        plot(a[sq], CL[sq], '-o')
        title(f"Squadra {sq+1}, Coefficiente di portanza")
        xlabel("alpha [deg]")
        ylabel("$c_L$")
        grid()
        savefig(f'images/cl{sq+1}')
        close()
    
    for sq in range(4):
        plot(a[sq], CL[sq], '-o')
    title(f"Coefficiente di portanza")
    xlabel("alpha [deg]")
    ylabel("$c_L$")
    grid()
    legend(["Squadra 1", "Squadra 2", "Squadra 3", "Squadra 4"])
    savefig(f'images/cl')
    close()


def save_cp_plots(x, cpd, cpv):
    for sq in range(4):
        for i in range(1, len(cpd[sq])):
            title(f"Squadra {sq+1}, alpha={cpd[sq][i][0]:.0f} deg")
            plot(x, cpd[sq][i][1:], '-o')
            plot(x, cpv[sq][i][1:], '-o')
            xlabel("x/c")
            ylabel("$c_p$")
            legend(["dorso", "ventre"])
            gca().invert_yaxis()
            grid()
            savefig(f'images/cp{sq+1} a={cpd[sq][i][0]:.0f}')
            close()

    for sq in range(4):
        for i in range(1, len(cpd[sq])):
            plot(x, cpd[sq][i][1:], '-o', label=f"alpha={cpd[sq][i][0]}deg", color="bgrcmyk"[i-1])
            plot(x, cpv[sq][i][1:], '-o', color="bgrcmyk"[i-1])
        title(f"Coefficiente di pressione, Squadra {sq+1}")
        xlabel("x/c")
        ylabel("$c_p$")
        legend()
        gca().invert_yaxis()
        grid()
        savefig(f'images/cp{sq+1}')
        close()


def import_data():
    x = array([0, 2.5, 5, 10, 20, 30, 40, 50, 60, 70, 80, 100])/100
    cpd, cpv = [[], [], [], []], [[], [], [], []]
    for sq in range(4):
        with open(f'cpcsv/Data(cpd{sq+1}).csv', newline='') as f:
            reader = csv.reader(f)
            for row in reader:
                cpd[sq].append(row)
            for row in range(1,len(cpd[sq])):
                for i in range(13):
                    cpd[sq][row][i] = float(cpd[sq][row][i])
        with open(f'cpcsv/Data(cpv{sq+1}).csv', newline='') as f:
            reader = csv.reader(f)
            for row in reader:
                cpv[sq].append(row)
            for row in range(1,len(cpv[sq])):
                for i in range(13):
                    cpv[sq][row][i] = float(cpv[sq][row][i])
    return x, cpd, cpv
    

if __name__ == "__main__":
    main()
\end{lstlisting}

\subsection{Studio della scia del profilo alare Naca 0015}\label{b6}
\begin{lstlisting}
    from numpy import *
    from matplotlib.pyplot import *
    from scipy.interpolate import interp1d
    
    
    def main():
        c = 0.1
        Re = [75000, 100000, 90000, 120000]
    
        data50 = loadtxt("xfoil50k.txt")
        alpha50 = data50[:,0]
        CDx50 = data50[:,2]
        CLx50 = data50[:,1]
        plot(alpha50, CDx50, label="XFOIL Re=50000", linewidth=1)
    
        data100 = loadtxt("xfoil100k.txt")
        alpha100 = data100[:,0]
        CDx100 = data100[:,2]
        CLx100 = data100[:,1]
        plot(alpha100,CDx100, label="XFOIL Re=100000", linewidth=1)
        title("Coefficiente di resistenza (XFOIL)")
        ylabel("$c_D$")
        xlabel(r"$\alpha$ [$^\circ$]")
        legend()
        grid()
        axis([-15.5, 15.5, 0.014, 0.12])
        savefig("xfoil", dpi=400)
        close()
        
        CDD = []
    
        for sq in range(4):
            a, y, uOvV = import_data(sq)
    
            CD = zeros(len(a))
            for i in range(len(a)):
                CD[i] = 2/c * trapz(uOvV[i,:] * (1-uOvV[i,:]), y)
            CDD.append(CD)
            plot(concatenate((-a[::-1], a)), concatenate((CD[::-1], CD)), '-o', linewidth=1, markersize=3, label=f"Sq. {sq+1}, Re: {Re[sq]}")
        grid()
        axis([-15.5, 15.5, 0.014, 0.12])
        title("Coefficiente di resistenza")
        ylabel("$c_D$")
        xlabel(r"$\alpha$ [$^\circ$]")
        legend()
        savefig("cd", dpi=400)
        close()
    
        for sq in range(4):
            Deltaw = []
            a, y, uOvV = import_data(sq)
            for alpha in range(len(a)):
                figure(0)
                plot(y, uOvV[alpha, :], label=r"$\alpha$"+f"={int(a[alpha])}"+r"$^\circ$")
                ymin = y[argmin(uOvV[alpha, :])]
                y90 = interp1d(uOvV[alpha, :], y, fill_value="extrapolate")(0.9*max(uOvV[alpha, :]))
                Deltaw.append(abs(2 * (y90 - ymin)))
            legend()
            title(f"Squadra {sq+1}")
            xlabel("$y$ [m]")
            ylabel(r"$\frac{u}{V_\infty}$")
            grid()
            savefig(f"v{sq+1}")
            close()
    
            figure(1)
            plot(a, Deltaw, label=f"Squadra {sq+1}")
        title("Dimensione trasversale della scia")
        xlabel(r"$\alpha$ [$^\circ$]")
        ylabel(r"$\Delta_w(\alpha)$ [m]")
        legend()
        grid()
        savefig(f"dim scia", dpi=400)
        close()
    
    
        CL = []
        CL.append([0.0, 0.21337719276250003, 0.39920634915000003, 0.5718014705999999, 0.6973437499999999, 0.8406250000000001, 0.6942023024999999])
        CL.append([0.0, 0.2541666667875, 0.38419117663749996, 0.5542929292000001, 0.7519700460625001, 0.9096536999874999, 0.749770021375])
        CL.append([0.0, 0.2407407410375, 0.457314560575, 0.6075206043, 0.7551504628000001, 0.8209433232375001, 0.764903846275])
        CL.append([0.0, 0.1705337691125, 0.4198005699124999, 0.5916666666125001, 0.8583450423625, 0.9174688057000001, 0.9184906124625001])
    
        for i in range(4):
            CL[i] = array(CL[i])
            x = concatenate((CDD[i][:6], [CDD[i][-1]]))
            plot(concatenate((x[::-1], x)), concatenate((-CL[i][::-1], CL[i])), label=f"Sq. {i+1}, Re: {Re[i]}")
        grid()
        title("Polare aerodinamica")
        legend()
        xlabel("$c_D$")
        ylabel("$c_L$")
        axis([0.015, 0.12, -1.2, 1.2])
        savefig("clvcd", dpi=400)
        close()
    
        plot(CDx50, CLx50, label="XFOIL Re=50000")
        plot(CDx100, CLx100, label="XFOIL Re=100000")
        grid()
        title("Polare aerodinamica")
        legend()
        xlabel("$c_D$")
        ylabel("$c_L$")
        axis([0.015, 0.12, -1.2, 1.2])
        savefig("clvcdxfoil", dpi=400)
        close()
    
    
    def import_data(sq):
        raw = loadtxt(open(f"csv/Data(sq{sq+1}).csv", "rb"), delimiter=",")
        alpha = raw[:,0]
        uOvV = raw[:,1:]
        y = linspace(0, 0.034, 18)
        return alpha, y, uOvV
    
    
    if __name__ == "__main__":
        main()
\end{lstlisting}

\subsection{Flusso in un condotto piano}\label{b7}
\begin{lstlisting}
from numpy import *
from matplotlib.pyplot import *
from glob import glob


def main():
    H, B = 0.07, 0.3
    D = 4*B*H/(2*(B+H))
    x = [0, 0.38, 0.76, 1.185, 1.401, 1.785, 2.17, 2.59, 2.82, 3.19, 3.8]
    pamb = array([101800, 101700, 101600, 101300])
    T = array([20.2, 22, 22.1, 21.2]) + 273.15
    rho = pamb / 287 / T
    mu = 1.46E-6 * T**1.5 / (T + 110)

    for sq in range(4):
        file_paths = glob(f"sq{sq+1}/n*.dsa")
        n, ptot, ps, p = [], [], [], []
        for file_path in file_paths:
            n.append(float(file_path.split("n")[1].split("_")[0]))
            data = genfromtxt(file_path, skip_header=5).T
            ptot.append(average(data[1]))
            ps.append(average(data[2]))
            p.append(array([average(data[i]) for i in range(3,14)]))
        n, ptot, ps, p = zip(*sorted(zip(n, ptot, ps, p)))
        n, ptot, ps, p = array(n), list(ptot), list(ps), list(p)

        if sq == 1:
            for i in range(len(n)):
                p[i][3] = (p[i][2] + p[i][4])/2

        q = array([ptot[i] - ps[i] for i in range(len(ptot))])*0.7225
        u = sqrt(2*q/rho[sq])
        Re = rho[sq] * u * D / mu[sq]
        dpdx = array([polyfit(x, p[i], 1)[0] for i in range(len(n))])
        lambd = -dpdx * D / q

        figure(0)
        plot(Re, dpdx, "o", label=f"Squadra {sq+1}")

        figure(1)
        plot(Re, lambd, "o", label=f"Squadra {sq+1}")

        figure(2)
        loglog(Re, lambd, "o", label=f"Squadra {sq+1}")

    figure(0)
    xlabel("Reynolds")
    ylabel(r"$\frac{dp}{dx}$ [Pa/m]")
    legend()
    grid()
    tight_layout()
    savefig("images/dpdx", dpi=400)
    close()

    lam = lambda _Re: array([96/Re for Re in _Re])
    turb = lambda _Re: array([0.32/Re**0.25 for Re in _Re])

    figure(1)
    Re_range = linspace(1600, 6000, 100)
    plot(Re_range, lam(Re_range), label="Legge laminare")
    Re_range = linspace(6000, 25000, 100)
    plot(Re_range, turb(Re_range), label="Legge turbolenta")
    xlabel("Reynolds")
    ylabel(r"$\lambda$")
    legend()
    tight_layout()
    grid(which='both')
    savefig("images/lambda", dpi=400)
    close()

    figure(2)
    Re_range = linspace(2000, 6000, 100)
    loglog(Re_range, lam(Re_range), label="Legge laminare")
    Re_range = linspace(6000, 50000, 100)
    loglog(Re_range, turb(Re_range), label="Legge turbolenta")
    xlabel("Reynolds")
    ylabel(r"$\lambda$")
    legend()
    tight_layout()
    grid(which='both')
    savefig("images/lambdaloglog", dpi=400)
    close()

    figure(3)
    plot(x, p[5], "o")
    grid()
    title(f"Squadra 4, Re={Re[5]:.0f}")
    xlabel("x [m]")
    ylabel(r"$\Delta p$ [Pa]")
    savefig("images/p", dpi=400)
    close()

    close("all")


if __name__ == "__main__":
    main()
\end{lstlisting}

\subsection{Taratura di una sonda a filo caldo}\label{b8}
\begin{lstlisting}
from numpy import *
from matplotlib.pyplot import *


def main():
    raw, Et, U, E, x, y, A, B, n, Uk, U2, U3, U4, U5  = ([[], [], [], []] for _ in range(14))
    a2, b2, c2 = ([[], [], [], []] for _ in range(3))
    a3, b3, c3, d3 = ([[], [], [], []] for _ in range(4))
    a4, b4, c4, d4, e4 = ([[], [], [], []] for _ in range(5))
    a5, b5, c5, d5, e5, f5 = ([[], [], [], []] for _ in range(6))
    epsk, eps2, eps3, eps4, eps5 = ([[], [], [], []] for _ in range(5))
    E0t, E0 = zeros(4), zeros(4)

    K = 550
    p = array([735, 735, 735, 734])/760*101325
    T = array([21, 20.8, 21.8, 20.8]) + 273.15
    rho = p/287/T

    for sq in range(4):
        # Plot dati grezzi
        raw[sq] = loadtxt(open(f"Misure/DatiCalibrazione_Sq_{sq+1}.txt", "rb"), skiprows=1)
        Et[sq] = raw[sq][:, 0]
        E[sq] = raw[sq][:, 1]
        E0t[sq] = Et[sq][0]
        E0[sq] = E[sq][0]
        Et[sq] -= E0t[sq]
        U[sq] = sqrt(2 / rho[sq] * K * Et[sq])
        figure(4)
        plot(U[sq], E[sq], '-o', label=f"Squadra {sq + 1}")
        grid(True)
        legend()
        figure(sq)
        plot(U[sq], E[sq], 'o')
        grid()
        title(f"Squadra {sq+1}")

        # Legge di King
        y[sq] = log10(E[sq][1:]**2 - E0[sq]**2)
        x[sq] = log10(U[sq][1:])
        A[sq] = E0[sq]**2
        n[sq], B[sq] = polyfit(x[sq], y[sq], 1)
        B[sq] = 10**B[sq]
        # print(A[sq], B[sq], n[sq])

        # Plot legge di King
        Uk[sq] = lambda E: ((E**2 - A[sq])/B[sq])**(1/n[sq])
        figure(sq)
        plot(Uk[sq](linspace(E[sq][0], E[sq][-1])), linspace(E[sq][0], E[sq][-1]), label=f"King")

        # Legge polinomiale quadratica
        c2[sq], b2[sq], a2[sq] = polyfit(E[sq], U[sq], 2)
        U2[sq] = lambda E: a2[sq] + b2[sq] * E + c2[sq] * E**2
        figure(sq)
        plot(U2[sq](linspace(E[sq][0], E[sq][-1])), linspace(E[sq][0], E[sq][-1]), label=f"Quadratica")

        # Legge polinomiale cubica
        d3[sq], c3[sq], b3[sq], a3[sq] = polyfit(E[sq], U[sq], 3)
        U3[sq] = lambda E: a3[sq] + b3[sq] * E + c3[sq] * E**2 + d3[sq] * E**3
        figure(sq)
        plot(U3[sq](linspace(E[sq][0], E[sq][-1])), linspace(E[sq][0], E[sq][-1]), label=f"Cubica")

        # Legge polinomiale di quarto grado
        e4[sq], d4[sq], c4[sq], b4[sq], a4[sq] = polyfit(E[sq], U[sq], 4)
        U4[sq] = lambda E: a4[sq] + b4[sq] * E + c4[sq] * E**2 + d4[sq] * E**3 + e4[sq] * E**4
        figure(sq)
        plot(U4[sq](linspace(E[sq][0], E[sq][-1])), linspace(E[sq][0], E[sq][-1]), label=f"Quarto grado")

        # Legge polinomiale di quinto grado
        f5[sq], e5[sq], d5[sq], c5[sq], b5[sq], a5[sq] = polyfit(E[sq], U[sq], 5)
        U5[sq] = lambda E: a5[sq] + b5[sq] * E + c5[sq] * E**2 + d5[sq] * E**3 + e5[sq] * E**4 + f5[sq] * E**5
        figure(sq)
        plot(U5[sq](linspace(E[sq][0], E[sq][-1])), linspace(E[sq][0], E[sq][-1]), label=f"Quinto grado")

        # Save figures
        legend()
        xlabel("$U_{pitot}$")
        ylabel("$E_{HW}$")
        savefig(f'images/sq{sq+1}')

        # Stimatori
        epsk[sq] = sqrt(sum(((Uk[sq](E[sq][1:])-U[sq][1:])/Uk[sq](E[sq][1:]))**2))
        eps2[sq] = sqrt(sum(((U2[sq](E[sq][1:])-U[sq][1:])/U2[sq](E[sq][1:]))**2))
        eps3[sq] = sqrt(sum(((U3[sq](E[sq][1:])-U[sq][1:])/U3[sq](E[sq][1:]))**2))
        eps4[sq] = sqrt(sum(((U4[sq](E[sq][1:])-U[sq][1:])/U4[sq](E[sq][1:]))**2))
        eps5[sq] = sqrt(sum(((U5[sq](E[sq][1:])-U[sq][1:])/U5[sq](E[sq][1:]))**2))
        print(epsk[sq], eps2[sq], eps3[sq], eps4[sq], eps5[sq])

    print(average(A), average(B), average(n))

    figure(4)
    xlabel(r"$U_{pitot}$")
    ylabel(r"$E_{HW}$")
    savefig(f"images/rawdata")


if __name__ == "__main__":
    main()
\end{lstlisting}

\subsection{Strato limite su placca piana}\label{b9}
\begin{lstlisting}
from glob import glob
from pandas import *
from numpy import *
from matplotlib.pyplot import *
from scipy.optimize import *
from scipy.interpolate import interp1d
from scipy.signal import welch


def main():
    A, B, n = 2.02849375, 0.6230599504639609, 0.5441415727388867
    L, Height = 1.45, 0.5
    blasius = loadtxt("Blasius.csv", delimiter=",")
    eta_b, phi_b = blasius[:,0], blasius[:,1]
    rho = 101100 / 287 / 293
    mu = 1.46E-6 * 293**1.5 / (293 + 110)
    nu = mu / rho
    X = array([[0.55, 0.9],[0.5, 0.9],[0.9, 0.9],[0.7, 0.7]])
    FS = [30000, 30000, 30000, 20000]

    for sq in range(1,5):
        for profilo in range(1,3):
            file_paths = glob(f'Misure/Dati_Sq{sq}/Profilo{profilo}/*.csv')
            E_i, y = [], []
            for file_path in file_paths:
                y.append(float(file_path.split('_')[2].split('.csv')[0]))
                E_i.append(read_csv(file_path, header=None))
            y = array(y)
            if sq == 1:
                y += 0.5
            E = hstack(E_i)
            sorted_indices = argsort(y)
            E = E[:, sorted_indices]
            y = y[sorted_indices]
            cf = None

            # Calcolo delle velocita'
            # A = average(E[:,0])**2
            U = sign(E**2-A)*abs(((E**2-A)/B))**(1/n)
            u = array([average(U[:,i]) for i in range(U.shape[1])])
            u_rms = array([std(U[:,i]) for i in range(U.shape[1])])

            # Plot grandezze dimensionali
            figure()
            plot(u, y)
            title(f"Squadra {sq} Profilo {profilo}")
            xlabel("u [m/s]")
            ylabel("y [mm]")
            grid()
            savefig(f'images/sq{sq}p{profilo}', dpi=400)
            close()

            # Calcolo proprieta' dello strato limite
            Ue = max(u)
            delta99 = interp1d(u, y, fill_value="extrapolate")(0.99*Ue)
            deltastar = trapz(1-u/Ue, y)
            theta = trapz(u/Ue*(1-u/Ue), y)
            H = deltastar / theta

            # Plot grandezze adimensionali
            figure()
            plot(u/Ue, y/delta99)
            title(f"Squadra {sq} Profilo {profilo}")
            xlabel(r"$\frac{u}{U_e}$")
            ylabel(r"$\frac{y}{\delta_{99}}$")
            grid()
            savefig(f'images/sq{sq}p{profilo}_adim', dpi=400)
            close()

            # Plot RMS adimensionali
            figure()
            plot(u_rms/Ue, y/delta99)
            title(f"Squadra {sq} Profilo {profilo}")
            xlabel(r"$\frac{u_{rms}}{U_e}$")
            ylabel(r"$\frac{y}{\delta_{99}}$")
            grid()
            savefig(f'images/sq{sq}p{profilo}_rms_adim', dpi=400)
            close()

            # Calcolo del numero di Reynolds
            x = X[sq-1,profilo-1]
            Rex = Ue*x/nu

            if Rex < 500000:
                # Confronto con Blasius in caso laminare
                eta = y/1000/sqrt(nu*x/Ue)
                phi = u/Ue

                figure()
                plot(phi, eta, label="Valori sperimentali")
                plot(phi_b, eta_b, label="Blasius")
                title(f"Squadra {sq} Profilo {profilo}")
                xlabel(r"$\phi$")
                ylabel(r"$\eta$")
                legend()
                grid()
                savefig(f'images/sq{sq}p{profilo}_blasius', dpi=400)
                close()

            if Rex > 250000: 
                # Metodo di Clauser
                x_clauser = (y/1000 * Ue / nu)[1:]
                y_clauser = (u / Ue)[1:]
                clauser = lambda x, cf: sqrt(cf/2)*(log(x * sqrt(cf/2))/0.41 + 5.2)
                cf = fmin(lambda cf: sum(abs(clauser(x_clauser, cf) - y_clauser)), 0.001, disp=0)[0]
                tau_w = rho * Ue**2 * cf / 2

                figure()
                for i in range(8):
                    semilogx(logspace(2, 5), clauser(logspace(2, 5), 0.0005 + i*0.0005), 'c')
                semilogx(x_clauser, y_clauser, 'o')
                title(f"Squadra {sq} Profilo {profilo}")
                xlabel(r"$\frac{y\ U_e}{\nu}$")
                ylabel(r"$\frac{u}{U_e}$")
                grid()
                savefig(f'images/sq{sq}p{profilo}clauser', dpi=400)
                close()

                # Confronto con Van Driest in caso turbolento
                u_tau = sqrt(tau_w / rho)
                y_plus = y/1000 * u_tau / nu
                u_plus = u / u_tau
                u_rms_plus = u_rms / u_tau

                y_vd_visc = linspace(0, 5, 6)
                u_vd_visc = y_vd_visc
                y_vd_log = linspace(30, y_plus[-1])
                u_vd_log = log(y_vd_log)/0.41 + 5.2
                y_vd = concatenate((y_vd_visc, y_vd_log))
                u_vd = concatenate((u_vd_visc, u_vd_log))

                # Plot grandezze turbolente
                figure()
                plot(u_plus, y_plus, label="Dati sperimentali")
                plot(u_vd, y_vd, label="Van Driest")
                title(f"Squadra {sq} Profilo {profilo}")
                xlabel("$u^+$")
                ylabel("$y^+$")
                legend()
                grid()
                savefig(f'images/sq{sq}p{profilo}+', dpi=400)
                close()

                # Plot RMS turbolenti
                figure()
                plot(u_rms_plus, y_plus)
                title(f"Squadra {sq} Profilo {profilo}")
                xlabel("$u^+_{rms}$")
                ylabel("$y^+$")
                grid()
                savefig(f'images/sq{sq}p{profilo}+_rms', dpi=400)
                close()

            # Stampa risultati
            print(f"Squadra {sq}, Profilo {profilo}")
            print(f"Reynolds: {Rex:.0f}")
            print(f"Ue \t= {Ue:.2f} m/s")
            print(f"x \t= {x} m")
            if cf is not None:
                print(f"cf \t= {cf:.5f}")
                print(f"tau_w \t= {tau_w:.2f} N/m2")
            print(f"delta99 \t= {delta99:.2f} mm")
            print(f"delta* \t= {deltastar:.2f} mm")
            print(f"theta \t= {theta:.2f} mm")
            print(f"H \t= {H:.2f}\n")

            # Analisi temporale
            E = loadtxt(glob(f'Misure/Dati_Sq{sq}/timeseries*.csv')[0])
            U = sign(E**2-A)*abs(((E**2-A)/B))**(1/n)
            fs = FS[sq-1]
            Nx = len(U)
            T = Nx / fs
            t = linspace(0, T - 1/fs, Nx)

            # Plot velocita' nel tempo
            figure()
            plot(t, U, linewidth=0.012)
            title(f"Squadra {sq}")
            xlabel("t [s]")
            ylabel("u [m/s]")
            grid()
            savefig(f'images/sq{sq}timeseries', dpi=400)
            close()
            
            # Parametri per lo spettro di potenza
            nsc = int(floor(Nx / 4.5))
            nov = int(floor(nsc / 2))
            nff = max(256, 2**int(ceil(log2(nsc))))
            f, Pxx = welch(U - average(U), fs, 'hann', nsc, nov, nff)

            # Plot spettro di potenza
            figure()
            loglog(f, Pxx, linewidth=0.5)
            title(f"Squadra {sq}")
            xlabel("f [Hz]")
            ylabel("Densita' spettrale di potenza")
            grid()
            savefig(f'images/sq{sq}timeserieswelch', dpi=400)
            close()


if __name__ == "__main__":
    main()
\end{lstlisting}

\subsection{Vortex shedding di un cilindro}\label{b10}
\begin{lstlisting}
import pandas as pd
from numpy import *
from matplotlib.pyplot import *
from scipy.signal import welch


def main():
    E0, E, f = import_data()
    p = array([740, 738, 737, 755]) / 760 * 101325
    T = array([19, 24, 24, 23]) + 273.15
    rho = p / T / 287
    mu = 1.46E-6 * T**1.5 / (T + 110)
    D = 1E-3

    for sq in range(4):
        E[sq] -= E0[sq]
        U = sqrt(2 * E[sq] * 540 / rho[sq])

        Re = rho[sq] * U * D / mu[sq]
        St = f[sq] * D / U

        figure()
        plot(U, f[sq], "o")
        title(f"Squadra {sq+1}")
        xlabel(r"$U_\infty$ [m/s]")
        ylabel(r"$f_{shedding}$ [Hz]")
        grid()
        savefig(f'images/sq{sq+1}', dpi=400)
        close()

        figure()
        plot(Re, St, "o", label="Dati sperimentali")
        plot(linspace(50, max(Re) + 10), St_t(linspace(50, max(Re) + 10)), label="Leggi empiriche")
        title(f"Squadra {sq+1}")
        xlabel("Reynolds")
        ylabel("Strouhal")
        legend()
        grid()
        savefig(f'images/sq{sq+1}adim', dpi=400)
        close()

        figure(0)
        plot(Re, St, "o", label=f"Squadra {sq+1}")

    plot(linspace(50, 1000), St_t(linspace(50, 1000)), label="Leggi empiriche")
    xlabel("Reynolds")
    ylabel("Strouhal")
    legend()
    grid()
    savefig(f'images/adim', dpi=400)
    close()

    U = []
    f_samp = 10000
    with open("timeseries.txt", 'r') as f:
        for line in f:
            try:
                U.append(float(line.strip().split()[2]))
            except:
                pass
    U = array(U)
    t = linspace(0, (len(U)-1)/f_samp, len(U))

    Uavg = average(U)
    Urms = std(U)
    print(f"Velocita' media: {Uavg:.2f} m/s. rms: {Urms:.3f} m/s")

    nsc = int(floor(len(U) / 4.5))
    nov = int(floor(nsc / 2))
    nff = max(256, 2**int(ceil(log2(nsc))))
    f, Pxx = welch(U - Uavg, f_samp, 'hann', nsc, nov, nff)

    plot(t, U, linewidth=0.2)
    xlabel("t [s]")
    ylabel("U [m/s]")
    grid()
    savefig(f'images/timeseries', dpi=400)
    close()

    loglog(f, Pxx, linewidth=0.5)
    xlabel("f [Hz]")
    ylabel("Densita' spettrale di potenza")
    grid()
    savefig(f'images/welch', dpi=400)
    close()

    print(f"Frequenza di sfilamento: {f[argmax(Pxx)]:.2f} Hz")
    

def St_t(_Re):
    St = []
    for Re in _Re:
        if Re == 50:
            St.append(0)
        elif 50 < Re <= 150:
            St.append(0.212 * (1-21.2/Re))
        elif 150 < Re <= 300:
            St.append(0.212 * (1-17/Re))
        elif 300 < Re <= 2000:
            St.append(0.212 * (1-12.7/Re))
    return St


def import_data():
    file_path = "Misure.xlsx"
    excel_file = pd.ExcelFile((file_path))
    sheet_names = excel_file.sheet_names
    dfs = [pd.read_excel(file_path, sheet_name=sheet) for sheet in sheet_names]
    s, E0, E, f = [], [], [], []
    for sq in range(4):
        s.append(dfs[sq].values.tolist())
        E0.append(s[sq][0][0])
        E.append(array([s[sq][x][0] for x in range(1, len(s[sq]))]))
        f.append(array([s[sq][x][1] for x in range(1, len(s[sq]))]))
    return E0, E, f


if __name__ == "__main__":
    main()
\end{lstlisting}

\subsection{PIV: studio della scia di corpi tozzi}\label{b11}
Per l'analisi dati della presente esercitazione sono stati utilizzati i seguenti codici in Matlab:
\begin{lstlisting}[style=Matlab-editor, basicstyle=\ttfamily\footnotesize]
% Codice per generare i fotogrammi delle animazioni
close all
clear
clc

filename = "cilindro22.mat";
disp("Loading data...")
load(filename)
frames = length(u_original);
fps = 60;
frameWidth = 1080;

%%
screenSize = get(0, 'ScreenSize');
screenWidth = screenSize(3);
screenHeight = screenSize(4);

M(frames) = struct('cdata',[],'colormap',[]);

x_original = x;
y_original = y;

x_min = min(cellfun(@(x) min(x(:)), x_original));
x_max = max(cellfun(@(x) max(x(:)), x_original));
y_min = min(cellfun(@(y) min(y(:)), y_original));
y_max = max(cellfun(@(y) max(y(:)), y_original));

u_max = max(cellfun(@(u) max(abs(u(:))), u_original));
v_max = max(cellfun(@(v) max(abs(v(:))), v_original));
V_max = sqrt(u_max^2 + v_max^2);

h = figure("Visible", "off");
box on
xlabel("x [m]")
ylabel("y [m]")
frameHeight = frameWidth*(y_max-y_min)/(x_max-x_min);
left = (screenWidth - frameWidth) / 2;
bottom = (screenHeight - frameHeight) / 2;
set(h, "Position", [left, bottom, frameWidth, frameHeight]);

for frame=1:frames
    u = u_original{frame};
    v = v_original{frame};
    xf = x{frame};
    yf = y{frame};
    Vmag = sqrt(u.^2+v.^2);
    
    imagesc(xf(1,:),yf(:,1),Vmag,'Interpolation','bilinear');
    hold on
    clim([0 V_max*0.7])
    axis([x_min x_max y_min y_max]);
    colormap(jet)
    clb=colorbar;
    title(clb,'V [m/s]')
    set(streamslice(xf,yf,u,v,0.3),'Color','k')
    quiver(xf,yf,u/90,v/90, "off", 'Color','k')
    hold off

    drawnow
    M(frame) = getframe;
    disp("Frame " + frame + " completed")
end
disp("Saving progress...")
save(filename)

%%
close all
h = figure();
box on
xlabel("x [m]")
ylabel("y [m]")
clim([0 V_max*0.7])
axis([x_min x_max y_min y_max]);
colormap(jet)
clb=colorbar;
title(clb,'V [m/s]')
set(h, "Position", [left, bottom, frameWidth, frameHeight]);
disp("Playing animation...")
movie(M, 10, fps);
\end{lstlisting}

\begin{lstlisting}[style=Matlab-editor, basicstyle=\ttfamily\footnotesize]
% codice utilizzato per l'analisi dati
close all
clear
clc

anno = 222;

disp("Loading data...")
if anno == 22
    filename = "PIV/2022/cilindro_22_600.mat";
    ix = 14;
    iy = 17;
    ix_inf = 16; 
    iy_inf = 17; 
elseif anno == 23
    filename = "PIV/2023/cilindro_23_600.mat";
    ix = 20; 
    iy = 11; 
    ix_inf = 26;
    iy_inf = 5;
elseif anno == 222
    filename = "cilindro22.mat";
    ix = 14;
    iy = 17;
    ix_inf = 16; 
    iy_inf = 17; 
else
    filename = "PIV/2022/placca_22_600.mat";
    ix = 14;
    iy = 17;
    ix_inf = 3; 
    iy_inf = 3; 
end
load(filename)
frames = length(u_original);

% NaN FIX
for frame = 1:frames
    mat = u_original{frame};
    mat(isnan(mat)) = 0;
    u_original{frame} = mat;
    mat = v_original{frame};
    mat(isnan(mat)) = 0;
    v_original{frame} = mat;
end

% Calcolo della velocita' in un punto e della velocita' a monte
u = zeros(frames, 1);
u_inf = zeros(frames, 1);
for frame=1:frames
    u_inf(frame) = sqrt(u_original{frame}(iy_inf, ix_inf)^2 + v_original{frame}(iy_inf, ix_inf)^2);
    u(frame) = sqrt(v_original{frame}(iy, ix)^2 + u_original{frame}(iy, ix)^2);
end

k = figure();
plot(linspace(0, frames/fps, frames), u)
xlabel("t [s]")
ylabel("u [m/s]")
grid minor
print(k, sprintf('images/timeseries%d.png', anno), '-dpng', '-r300');

% Calcolo della trasformata di Fourier
Fs = fps;
Nx = length(u);
f = (0:floor(Nx/2)) * (Fs/Nx);

q = figure();
U = fft(u - mean(u));
hold on
loglog(f, abs(real(U(1:floor(Nx/2)+1))));
loglog(f, abs(imag(U(1:floor(Nx/2)+1))));
legend("Parte reale", "Parte immaginaria")
xlabel('f [Hz]')
ylabel('Fast Fourier Transform [m/s]')
grid on
print(q, sprintf('images/FFT%d.png', anno), '-dpng', '-r300');

% Calcolo dello spettro di potenza
j = figure();
Pxx = (1/(Fs*Nx)) * abs(U).^2;
Pxx = Pxx(1:floor(Nx/2)+1);
loglog(f, Pxx);

xlabel('f [Hz]')
ylabel('Power Spectral Density (PSD) [J/kg]')
grid on
print(j, sprintf('images/PSD%d.png', anno), '-dpng', '-r300');

[pxxmax, ind] = max(Pxx);
disp("Frequenza di shedding: " + f(ind) + " Hz")

U_inf = mean(abs(u_inf(isfinite(u_inf))));
disp("Velocita' a monte: " + U_inf)

% Calcolo dei numeri di Reynolds e Strouhal
if anno == 220
    D = 0.024;
else
    D = 0.01;
end
nu_w = 0.000931 / 1000;
Re = U_inf * D / nu_w;
St = f(ind) * D / U_inf;

disp("Numero di Reynolds: " + Re)
disp("Numero di Strouhal: " + St)

%% Avvia animazione
h = figure();
box on
xlabel("x [m]")
ylabel("y [m]")
clim([0 V_max*0.7])
axis([x_min x_max y_min y_max]);
colormap(jet)
clb=colorbar;
title(clb,'V [m/s]')
set(h, "Position", [left, bottom, frameWidth, frameHeight]);
disp("Playing animation...")
movie(M, 10, fps);

%% Campo istantaneo
frame = 326;
uist = u_original{frame};
vist = v_original{frame};
xf = x{frame};
yf = y{frame};
Vmag = sqrt(uist.^2+vist.^2);

hh = figure();
set(hh, "Position", [left, bottom, frameWidth, frameHeight]);
imagesc(xf(1,:),yf(:,1),Vmag,'Interpolation','bilinear');
hold on
% clim([0 max(Vmag)])
axis([x_min x_max y_min y_max]);
colormap(jet)
clb=colorbar;
title(clb,'V [m/s]')
quiver(xf,yf,uist, vist, 'Color','k')
print(hh, sprintf('images/f300_%b.png', anno), '-dpng', '-r300');

%% Campo medio e fluttuazioni
[nr, nc] = size(u_original{1});
% Campo medio
uavg = zeros(nr, nc);
vavg = zeros(nr, nc);
for frame = 1:frames
    uavg = uavg + u_original{frame};
    vavg = vavg + v_original{frame};
end
uavg = uavg/frames;
vavg = vavg/frames;
Vavg = sqrt(uavg.^2 + vavg.^2);
% Fluttuazioni
u_flutt = u_original;
v_flutt = v_original;
flutt = u_original;
for frame = 1:frames
    u_flutt{frame} = u_original{frame} - uavg;
    v_flutt{frame} = v_original{frame} - vavg;
    flutt{frame} = sqrt(u_flutt{frame}.^2 + v_flutt{frame}.^2);
end
% Teorema di Parseval
ix = 14;
iy = 17;
u = zeros(frames, 1);
for frame=1:frames
    u(frame) = sqrt(v_original{frame}(iy, ix)^2 + u_original{frame}(iy, ix)^2);
end
var = 0;
for frame = 1:frames
    var = var + (u(frame) - mean(u))^2;
end
var = var/frames;
pars = 2*trapz(linspace(0,30,300), Pxx);
disp("Varianza: " + var + " m2/s2")
disp("Integrale dello spettro di potenza: " + pars + " J/kgs")
% Sforzi di Reynolds
rho = 1000;
tauReist = u_original;
tauRe = zeros(nr, nc);
for frame = 1:frames
    tauReist{frame} = -rho .* u_flutt{frame} .* v_flutt{frame};
    tauRe = tauRe + tauReist{frame};
end
tauRe = tauRe/frames;

%% Genera immagine
frame = 326;
fig = figure();
set(fig, "Position", [left, bottom, frameWidth, frameHeight]);
imagesc(xf(1,:),yf(:,1), tauRe,'Interpolation','bilinear');
hold on
axis([x_min x_max y_min y_max]);
colormap(jet)
clb=colorbar;
% clim([0 0.08])
title(clb,'\tau_{Re} [N/m^2]')
% quiver(xf,yf,u_flutt{frame}, v_flutt{frame}, 'Color','k')
print(fig, "images/tauRe.png", '-dpng', '-r300');

%% Confronto Reynolds-Strouhal
close all
clear
clc

load("exported.mat")
load("exportedemp.mat")
L10emp(end,1) = 1400; 
z = figure();
hold on
plot(L10emp(:,1), L10emp(:,2))
plot(L10(:,1), L10(:,2), ".", 'MarkerSize', 15)
plot([1224 1296], [0.185 0.215], "k.", 'MarkerSize', 20)
legend("Relazioni empiriche", "Risultati E10", "Risultati PIV", 'Location', 'southeast')
xlabel("Reynolds")
ylabel("Strouhal")
grid on
print(z, "images/Re-St.png", '-dpng', '-r300');
\end{lstlisting}
