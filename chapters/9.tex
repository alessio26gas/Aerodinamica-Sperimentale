\section{Strato limite su placca piana}\label{c9}
La presente esercitazione si pone come obiettivo la caratterizzazione dello strato limite su placca piana mediante l'utilizzo dell'anemometria a filo caldo. In particolare, si vuole:
\begin{itemize}
    \item Misurare i profili di velocità per diverse ascisse $x$ con assegnata $U_\infty$;
    \item Caratterizzare la struttura dello strato limite:
    \begin{itemize}
        \item diagrammare i profili di velocità $u=u(x,y)$ e della deviazione standard delle fluttuazioni turbolente longitudinali $u_{rms}(x,y)$;
        \item valutare lo spessore geometrico $\delta(x)$, lo spessore di spostamento $\delta^*(x)$, lo spessore di quantità di moto $\theta(x)$ e il parametro di forma $H(x)$;
        \item verificare la condizione dello strato limite: laminare, transizionale o turbolento;
        \item diagrammare i profili di velocità media e fluttuante nella forma adimensionale: $u/U_\infty = f(y/\delta)$ e $u_{rms}/U_\infty = f(y/\delta)$.
    \end{itemize}
\end{itemize}

\subsection{Descrizione dell'esperimento}

\subsection{Catena di misura}

\subsection{Procedura sperimentale}

\subsection{Analisi dati}
